\section{Model Architecture}

We implemented three distinct CNN architectures for speech emotion recognition: a 1D CNN for spectral features, a 2D CNN for mel spectrogram images, and a combined model that leverages both feature spaces. This section describes the design of each architecture.

\subsection{1D CNN Architecture}

The 1D CNN is designed to process the time series of spectral features extracted from the audio signals. The architecture consists of three convolutional blocks followed by a global average pooling layer and a fully connected layer.

Each convolutional block in the 1D CNN includes:
\begin{itemize}
    \item 1D Convolutional layer with increasing filter sizes (128, 256, 512)
    \item Group Normalization for stable training
    \item Activation function (ReLU, SiLU, or ELU, depending on the experiment)
    \item Max Pooling layer with kernel size 2 and stride 2
    \item Dropout layer with dropout rate 0.3
\end{itemize}

\begin{lstlisting}[language=Python, style=pseudocode, caption=1D CNN Architecture (Pseudocode)]
class CNN1D:
    input: Spectral features with shape [batch_size, channels, time]
    
    ConvBlock1:
        Conv1D(in_channels=input_channels, out_channels=128, kernel_size=3, padding=1)
        GroupNorm(channels=128)
        Activation(ReLU/SiLU/ELU)
        MaxPool1D(kernel_size=2, stride=2)
        Dropout(p=0.3)
    
    ConvBlock2:
        Conv1D(in_channels=128, out_channels=256, kernel_size=3, padding=1)
        GroupNorm(channels=256)
        Activation(ReLU/SiLU/ELU)
        MaxPool1D(kernel_size=2, stride=2)
        Dropout(p=0.3)
    
    ConvBlock3:
        Conv1D(in_channels=256, out_channels=512, kernel_size=3, padding=1)
        GroupNorm(channels=512)
        Activation(ReLU/SiLU/ELU)
        MaxPool1D(kernel_size=2, stride=2)
        Dropout(p=0.3)
    
    GlobalAveragePooling1D()
    
    output: Feature vector of size 512
\end{lstlisting}

\subsection{2D CNN Architecture}

The 2D CNN processes mel spectrogram images and consists of four convolutional blocks followed by global average pooling and a fully connected layer.

Each convolutional block in the 2D CNN includes:
\begin{itemize}
    \item 2D Convolutional layer with increasing filter sizes (32, 64, 512, 1024)
    \item Group Normalization for stable training
    \item Activation function (ReLU, SiLU, or ELU, depending on the experiment)
    \item Max Pooling layer with kernel size (2,2) and stride 2
    \item Dropout layer with dropout rate 0.3
\end{itemize}

\begin{lstlisting}[language=Python, style=pseudocode, caption=2D CNN Architecture (Pseudocode)]
class CNN2D:
    input: Mel spectrogram with shape [batch_size, channels, height, width]
    
    ConvBlock1:
        Conv2D(in_channels=input_channels, out_channels=32, kernel_size=3x3, padding=1)
        GroupNorm(channels=32)
        Activation(ReLU/SiLU/ELU)
        MaxPool2D(kernel_size=2x2, stride=2)
        Dropout(p=0.3)
    
    ConvBlock2:
        Conv2D(in_channels=32, out_channels=64, kernel_size=3x3, padding=1)
        GroupNorm(channels=64)
        Activation(ReLU/SiLU/ELU)
        MaxPool2D(kernel_size=2x2, stride=2)
        Dropout(p=0.3)
    
    ConvBlock3:
        Conv2D(in_channels=64, out_channels=512, kernel_size=3x3, padding=1)
        GroupNorm(channels=512)
        Activation(ReLU/SiLU/ELU)
        MaxPool2D(kernel_size=2x2, stride=2)
        Dropout(p=0.3)
    
    ConvBlock4:
        Conv2D(in_channels=512, out_channels=1024, kernel_size=3x3, padding=1)
        GroupNorm(channels=1024)
        Activation(ReLU/SiLU/ELU)
        MaxPool2D(kernel_size=2x2, stride=2)
        Dropout(p=0.3)
    
    GlobalAveragePooling2D()
    
    output: Feature vector of size 1024
\end{lstlisting}

\subsection{Combined Model Architecture}

The combined model integrates both 1D and 2D feature paths. It processes spectral features using the 1D CNN path and mel spectrograms using the 2D CNN path, then concatenates the resulting feature vectors before passing them through a final fully connected layer for classification.

The combined model architecture:
\begin{itemize}
    \item Processes 1D spectral features through the 1D CNN path
    \item Processes 2D mel spectrograms through the 2D CNN path
    \item Concatenates the output feature vectors (512 features from 1D CNN, 1024 features from 2D CNN)
    \item Passes the combined features through a fully connected block with layer normalization and dropout
    \item Outputs emotion class probabilities
\end{itemize}

\begin{lstlisting}[language=Python, style=pseudocode, caption=Combined Model Architecture (Pseudocode)]
class CombinedModel:
    inputs:
        x_1d: Spectral features with shape [batch_size, channels, time]
        x_2d: Mel spectrogram with shape [batch_size, channels, height, width]
    
    # 1D CNN Path
    features_1d = CNN1D(x_1d)    # Output: [batch_size, 512]
    
    # 2D CNN Path  
    features_2d = CNN2D(x_2d)    # Output: [batch_size, 1024]
    
    # Feature Fusion
    combined_features = Concatenate([features_1d, features_2d])    # [batch_size, 1536]
    
    # Classification Head
    FCLayer:
        Linear(in_features=1536, out_features=128)
        LayerNorm(128)
        Activation(ReLU/SiLU/ELU)
        Dropout(p=0.5)
        Linear(in_features=128, out_features=num_classes)
    
    output: Class probabilities [batch_size, num_classes]
\end{lstlisting}

\subsection{Alternative Architectures Explored}

In addition to our main architecture, we also experimented with:

\begin{itemize}
    \item \textbf{ResNet-based models}: We implemented ResNet-38 and ResNet-101 architectures for comparison, which showed different levels of overfitting (particularly ResNet-101 with validation accuracy of 0.41 vs. train accuracy of 0.7)
    
    \item \textbf{Modified input dimensions}: We standardized the 2D input size to 64×64.
    
    \item \textbf{Variable length input}: Instead of fixed-size inputs, we also experimented with variable-length audio processing, which achieved better results (62.2\% accuracy on the combined model) compared to fixed-size inputs (61.3\%).
    
    \item \textbf{Modified 1D feature extraction}: We experimented with a single-channel approach using mean over time for fixed size input, which achieved comparable accuracy in the combined model.
\end{itemize}

These explorations helped us understand the trade-offs between model complexity, feature representation, and performance in the SER task. 